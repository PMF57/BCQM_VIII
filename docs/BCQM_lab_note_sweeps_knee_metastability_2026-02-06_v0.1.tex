\documentclass[11pt]{article}

\usepackage[a4paper,margin=25mm]{geometry}
\usepackage[T1]{fontenc}
\usepackage[utf8]{inputenc}
\usepackage[british]{babel}
\usepackage{lmodern}
\usepackage{microtype}
\usepackage{csquotes}
\usepackage{amsmath,amssymb}
\usepackage{booktabs}
\usepackage{enumitem}
\setlist{nosep}
\usepackage{xurl}
\usepackage{hyperref}
\usepackage[nameinlink,noabbrev]{cleveref}
\usepackage{graphicx}
\graphicspath{{./}{figures/}}

\title{BCQM Stage--3 Lab Note\\N-sweep and knee/clock-metastability study (v0.1)}
\author{Peter M.~Ferguson \\ \textit{Independent Researcher}}
\date{6 February 2026}

\begin{document}
\maketitle

\section*{Purpose}
Record and interpret a sequence of Stage--2/Stage--3 exploratory sweeps designed to (i) characterise scaling with ensemble size \(N\), (ii) locate the ``knee'' (core\(\rightarrow\)core+halo crossover), and (iii) test whether the time/ordering signal (\(Q_{\mathrm{clock}}\)) exhibits a mixed/metastable band near the crossover.

\section*{Runs executed (summary)}
All runs use \(n=0.8\), hits1 cloth extraction (min\_bin\_hits=1), \(W_{\mathrm{coh}}=100\), bins=20, x10 epoch, and 5 seeds unless otherwise noted.
\begin{itemize}
\item \textbf{Broad sweep (light):} \(N\in\{1,2,4,8,16,32,64,128\}\), tracing/ledger disabled; runtime \(\approx\) 50 minutes on Mac mini M4 (24 GB).
\item \textbf{Broad subset (heavy):} \(N\in\{32,64,128\}\), ledger+thread tracing enabled; runtime \(\approx\) 50 minutes (same wall time as light).
\item \textbf{Knee zoom (heavy):} \(N\in\{12,16,20,24,28,32,40,48\}\), ledger+thread tracing enabled; runtime \(\approx\) 100 minutes.
\item \textbf{Micro knee (heavy):} \(N\in\{21,22,23,24\}\), 5 seeds.
\item \textbf{Metastability confirmations (heavy, 20 seeds):} single-\(N\) runs at \(N=22\), \(N=23\), \(N=24\), and \(N=28\).
\end{itemize}

\section*{Broad scaling results}
\begin{table}[ht]
\centering
\begin{tabular}{@{}rcccc@{}}
\toprule
$N$ & $\Phi$ & $Q_{\mathrm{clock}}$ & core events & halo events \\
\midrule
1 & 1.0000 $\pm$ 0.0000 & 0.035 $\pm$ 0.032 & 3352 $\pm$ 58 & 0 $\pm$ 0 \\
2 & 1.0000 $\pm$ 0.0000 & 0.880 $\pm$ 0.143 & 6273 $\pm$ 73 & 0 $\pm$ 0 \\
4 & 1.0000 $\pm$ 0.0000 & 0.914 $\pm$ 0.195 & 11994 $\pm$ 66 & 0 $\pm$ 0 \\
8 & 1.0000 $\pm$ 0.0000 & 1.046 $\pm$ 0.231 & 23458 $\pm$ 137 & 0 $\pm$ 0 \\
16 & 0.9985 $\pm$ 0.0016 & 1.853 $\pm$ 1.982 & 46362 $\pm$ 148 & 69 $\pm$ 76 \\
32 & 0.1840 $\pm$ 0.0248 & 6.135 $\pm$ 0.207 & 17008 $\pm$ 2277 & 75417 $\pm$ 2401 \\
64 & 0.0747 $\pm$ 0.0003 & 8.816 $\pm$ 0.222 & 13730 $\pm$ 25 & 170121 $\pm$ 381 \\
128 & 0.0381 $\pm$ 0.0001 & 12.411 $\pm$ 0.313 & 13993 $\pm$ 12 & 352986 $\pm$ 649 \\
\bottomrule
\end{tabular}
\caption{Broad N-sweep summary (combined: light for $N\leq16$, heavy for $N\geq32$). Means and standard deviations are across 5 seeds per N.}
\label{tab:wide_sweep}
\end{table}

Figure~\ref{fig:phi_wide} and Figure~\ref{fig:q_wide} visualise the same trends. The broad sweep shows:
\begin{itemize}
\item a sharp core+halo transition between \(N=16\) and \(N=32\) in the powers-of-two scan (clarified by the knee zoom);
\item a rapidly decreasing core fraction \(\Phi\) for \(N\ge 32\) with a core size that saturates while the halo grows with \(N\);
\item an increasing and increasingly stable \(Q_{\mathrm{clock}}\) with \(N\), supporting the interpretation that ordered time becomes more robust as the ensemble scales.
\end{itemize}

\begin{figure}[ht]
  \centering
  \includegraphics[width=0.92\linewidth]{A_phi_vs_N_combined.pdf}
  \caption{Core fraction \(\Phi\) versus \(N\) (combined light/heavy).}
  \label{fig:phi_wide}
\end{figure}

\begin{figure}[ht]
  \centering
  \includegraphics[width=0.92\linewidth]{B_Qclock_vs_N_combined.pdf}
  \caption{Clock quality \(Q_{\mathrm{clock}}\) versus \(N\) (combined light/heavy).}
  \label{fig:q_wide}
\end{figure}

\section*{Knee localisation}
\begin{table}[ht]
\centering
\begin{tabular}{@{}rcccc@{}}
\toprule
$N$ & $\Phi$ & $Q_{\mathrm{clock}}$ & core events & halo events \\
\midrule
12 & 1.0000 $\pm$ 0.0000 & 1.480 $\pm$ 0.537 & 34831 $\pm$ 95 & 0 $\pm$ 0 \\
16 & 0.9990 $\pm$ 0.0007 & 3.021 $\pm$ 1.906 & 46388 $\pm$ 200 & 45 $\pm$ 33 \\
20 & 0.8708 $\pm$ 0.0445 & 3.918 $\pm$ 2.112 & 50297 $\pm$ 2535 & 7464 $\pm$ 2574 \\
24 & 0.4337 $\pm$ 0.0240 & 4.608 $\pm$ 1.833 & 30012 $\pm$ 1594 & 39189 $\pm$ 1772 \\
28 & 0.2552 $\pm$ 0.0206 & 5.740 $\pm$ 0.176 & 20588 $\pm$ 1647 & 60081 $\pm$ 1701 \\
32 & 0.1861 $\pm$ 0.0144 & 6.231 $\pm$ 0.058 & 17154 $\pm$ 1378 & 75015 $\pm$ 1148 \\
40 & 0.1187 $\pm$ 0.0051 & 6.850 $\pm$ 0.171 & 13679 $\pm$ 557 & 101604 $\pm$ 810 \\
48 & 0.0990 $\pm$ 0.0018 & 7.558 $\pm$ 0.067 & 13696 $\pm$ 273 & 124585 $\pm$ 292 \\
\bottomrule
\end{tabular}
\caption{Knee-zoom sweep (N=12–48). Heavy logging enabled; means and standard deviations across 5 seeds per N.}
\label{tab:knee_zoom}
\end{table}

The knee zoom tightens the location of the crossover: \(\Phi\) drops steeply between \(N=20\) and \(N=24\), with \(\Phi<0.5\) by \(N=24\). The core event count peaks around \(N\approx 20\) and then decreases, while the halo continues to grow, indicating a genuine rebalancing rather than simple halo accretion.

\section*{Micro knee (N=21--24)}
\begin{table}[ht]
\centering
\begin{tabular}{@{}rcccc@{}}
\toprule
$N$ & $\Phi$ & $Q_{\mathrm{clock}}$ & core events & halo events \\
\midrule
21 & 0.8008 $\pm$ 0.0493 & 2.386 $\pm$ 2.435 & 48426 $\pm$ 2886 & 12050 $\pm$ 3003 \\
22 & 0.6625 $\pm$ 0.0456 & 4.259 $\pm$ 1.980 & 42008 $\pm$ 2902 & 21398 $\pm$ 2886 \\
23 & 0.5427 $\pm$ 0.0258 & 5.232 $\pm$ 0.083 & 36080 $\pm$ 1740 & 30400 $\pm$ 1694 \\
24 & 0.4704 $\pm$ 0.0197 & 5.333 $\pm$ 0.086 & 32609 $\pm$ 1364 & 36712 $\pm$ 1374 \\
\bottomrule
\end{tabular}
\caption{Micro-sweep around the crossover (N=21–24). Heavy logging enabled; means and standard deviations across 5 seeds per N.}
\label{tab:knee_micro}
\end{table}

Figure~\ref{fig:phi_micro} shows that the \(\Phi=0.5\) crossing occurs between \(N=23\) and \(N=24\) in the 5-seed micro run.

\begin{figure}[ht]
  \centering
  \includegraphics[width=0.92\linewidth]{C_phi_knee_micro.pdf}
  \caption{Core fraction \(\Phi\) across the micro knee sweep (N=21--24).}
  \label{fig:phi_micro}
\end{figure}

\section*{Clock metastability (20-seed confirmations)}
\begin{table}[ht]
\centering
\begin{tabular}{@{}rrrrrcc@{}}
\toprule
$N$ & seeds & low-$Q$ & $p(\mathrm{low}\ Q)$ & $\Phi$ & $Q_{\mathrm{clock}}$ (all) & $Q_{\mathrm{clock}}$ (high band) \\
\midrule
22 & 20 & 5 & 0.25 & 0.6408 $\pm$ 0.0460 & 4.139 $\pm$ 1.870 & 5.177 $\pm$ 0.096 \\
23 & 20 & 4 & 0.20 & 0.5236 $\pm$ 0.0307 & 4.421 $\pm$ 1.806 & 5.274 $\pm$ 0.114 \\
24 & 20 & 2 & 0.10 & 0.4612 $\pm$ 0.0333 & 4.881 $\pm$ 1.424 & 5.336 $\pm$ 0.136 \\
28 & 20 & 0 & 0.00 & 0.2635 $\pm$ 0.0246 & 5.846 $\pm$ 0.143 & 5.846 $\pm$ 0.143 \\
\bottomrule
\end{tabular}
\caption{Clock metastability check (20-seed runs at fixed N). ``low-$Q$'' counts seeds with $Q_{\mathrm{clock}}<3$. The high-band statistics are computed over seeds with $Q_{\mathrm{clock}}\ge 3$.}
\label{tab:clock_meta}
\end{table}

The 20-seed confirmations establish that the mixed/metastable band is primarily a clock/ordering phenomenon:
\begin{itemize}
\item At \(N=22\), \(Q_{\mathrm{clock}}\) is clearly bimodal: a low-\(Q\) minority coexists with a tight high-\(Q\) locked band.
\item At \(N=23\) and \(N=24\), the low-\(Q\) minority persists but with decreasing prevalence.
\item By \(N=28\), no low-\(Q\) outliers were observed (0/20); \(Q_{\mathrm{clock}}\) is unimodal and tight, indicating robust lock-in.
\end{itemize}

\begin{figure}[ht]
  \centering
  \includegraphics[width=0.92\linewidth]{D_lowQ_fraction.pdf}
  \caption{Fraction of low-\(Q\) seeds (defined by \(Q_{\mathrm{clock}}<3\)) in the 20-seed confirmations.}
  \label{fig:lowq}
\end{figure}

\section*{Conclusions}
\begin{itemize}
\item The core\(\rightarrow\)core+halo crossover is sharp and lies in a narrow band around \(N\approx 22\)–24, with \(\Phi<0.5\) by \(N=24\).
\item The ordering/clock subsystem shows a mixed/metastable band in the same region: the probability of a low-\(Q\) outcome decreases with \(N\) and vanishes (in these samples) by \(N=28\).
\item Across the metastable band, the spatial cloth diagnostics remain broadly stable; the ``mixedness'' is primarily in the time/ordering variables (\(Q_{\mathrm{clock}}\), \(L\)).
\item Heavy logging (ledger+tracing) did not materially increase wall time relative to light runs on the reference machine, indicating runtime is dominated by the core simulation loop while disk usage becomes the main cost.
\end{itemize}

\end{document}
